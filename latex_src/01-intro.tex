\addchap{Введение}
Хакатон (от англ. hacker и marathon, «хакерский марафон») — соревнование для разработчиков, в котором команды на время создают прототип продукта, решающего ту или иную бизнес-задачу \cite{hackathon}.

Хакатоны в современном мире IT являются, массовым мероприятием, в котором принимают участие десятки, а порой и сотни людей \cite{leadersofdigital}. В связи с этим возникает необходимость наличия системы для хранения данных о мероприятии и его участниках, а также некоторой автоматизации 
 их обслуживания.

В данной работе ставится цель создания базы данных для хакатонов по направлению <<Big Data>>. 

Для достижения поставленной цели, необходимо решить следующие задачи:
\begin{itemize}
        \item проанализировать предметную область, выделить сущности и связи, которые должны моделироваться базой данных;
        \item проанализировать существующие модели данных;
        \item выбрать подходящую модель данных;
	\item формализовать задачу и определить необходимый функционал;
	\item описать структуру объектов БД, а также сделать выбор СУБД для ее хранения и взаимодействия;
	\item реализовать спроектированную БД;
	\item спроектировать и реализовать интерфейс для доступа к БД.
\end{itemize}

Итогом работы станет база данных с разработанным интерфейсом взаимодействия с ней с использованием различных ролевых моделей.