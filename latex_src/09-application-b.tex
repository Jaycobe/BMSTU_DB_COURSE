\addchap{Приложение Б}
Данное приложение содержит список разработанных для взаимодействия пользователя с базой данных HTTP-запросов.

\begin{itemize}
    \item POST (<</login>>) -- авторизация в приложении. При этом логин и пароль передаются в заголовке к запросу согласно методу BasicAuth\cite{basicauth};
    \item POST (<</logout>>) -- выход из учетной записи;
    \item GET (<</hackathons>>) -- получить данные о всех хакатонах;
    \item GET (<</hackathon/<id> >>) -- получить данные о хакатоне с идентификатором id;
    \item GET (<</hackathons/<id>/teams>>) -- получить данные о командах, 
    участвующих в хакатоне;
    \item GET (<</hackathons/my\_hackathons>>) -- получить данные о хакатонах текущего пользователя;
    \item POST (<</hackathons>>) -- создать хакатон. Доступно только администратору;
    \item DELETE (<</hackathons/<id> >>) -- удалить хакатон. Доступно только администратору и создателю хакатона;
    \item PUT (<</hackathons/<id> >>) -- редактировать данные о хакатоне. Доступно только администратору и создателю хакатона;
    \item GET (<</teams>>) -- получить данные о существующих командах;
    \item GET (<</teams/<id> >>) -- получить данные о команде с указанным идентификатором;
    \item GET (<</teams/my\_team>>) -- получить данные о команде, в которой состоит текущий пользователь;
    \item GET (<</teams/places\_available>>) -- получить список команд со свободными местами;
    \item GET (<</teams/<id>/participants>>) -- получить список участников команды;
    \item POST (<</teams>>) -- создать команду. Доступно только администратору;
    \item PUT (<</teams/<id>/leave>>) -- покинуть команду;
    \item PUT (<</teams/<team\_id>/set\_captain/>>) -- назначить нового капитана команды. Новым капитаном может быть только член команды. Доступно администратору и капитану команды;
    \item DELETE (<</teams/<team\_id>/<member\_id> >>) -- удалить из команды идентификатором team\_id участника с идентификатором member\_id. Доступно администратору и капитану команды;
    \item GET (<</participants>>) -- получить данные обо всех участниках;
    \item GET (<</participants/<id> >>) -- получить данные об участнике по его идентификатору;
    \item GET (<</organizers>>) -- получить данные обо всех организаторах;
    \item GET (<</organizers/<id> >>) -- получить данные об организаторе по его идентификатору;
    \item GET (<</requests>>) -- получить данные обо всех заявках. Доступно только администратору.
    \item GET (<</requests/open>>) -- получить данные об открытых заявках. Доступно только администратору.
    \item GET (<</requests/my\_requests) -- получить данные о заявках, которые подал текущий пользователь и которые назначены на него. Капитаны получают заявки о вступлении в команду, организаторы -- об участии команды в хакатоне, администраторы -- те заявки, которые они приняли;
    \item POST (<</requests>>) -- подать заявку;
    \item PUT (<</requests/accept/<id> >>) -- взять заявку в работу. Доступно только администратору;
    \item PUT (<</requests/approve/<id> >>) -- одобрить заявку. Капитаны могут одобрять заявки о вступлении в команду, организаторы -- об участии в хакатоне, администраторы -- могут совершать данное действие со всеми заявками;
    \item PUT (<</requests/decline/<id> >>) -- отклонить заявку.
\end{itemize}
