\addchap{Заключение}

В рамках курсового проекта была создана база данных для хакатонов по направлению <<Big Data>>.

Был проведён анализ существующих решений, предметной области, построена её модель в виде ER-диаграммы, выделены ролевые модели в системе, конкретизированы хранимые данные и выбрана модель базы данных.

Были формализованы сущности системы и спроектирован триггер AFTER на обновление рейтинга после присвоения призовых мест командам, участвовавшим в хакатоне. 

Была выбрана СУБД, средства реализации приложения, описано создание триггера, ролей и выделение им прав. Было представлено описание генерации данных для базы и примеры работы ПО. Было проведено нагрузочное тестирование по трём различным пользовательским сценариям, в ходе которого было выяснено, что разработанное приложение плохо справляется с большим количеством пользователей.

Был проведён эксперимент, в ходе которого было сравнение времени выполнения различных select запросов при наличии индексирования и без него. Полученные данные свидетельствуют о том, что индексирование способно до 200 раз ускорить выполнение запросов, однако имеет ограниченную применимость, замедляет массовую вставку в таблицу и требует времени для построение индекса над таблицей.  

Все поставленные задачи были выполнены: разработана база данных и интерфейс для взаимодействия с ней с использованием различных ролевых моделей. 